% Options for packages loaded elsewhere
\PassOptionsToPackage{unicode}{hyperref}
\PassOptionsToPackage{hyphens}{url}
%


\PassOptionsToPackage{table}{xcolor}

\documentclass[
  10pt,
  letterpaper,
]{article}

\usepackage{amsmath,amssymb}
\usepackage{iftex}
\ifPDFTeX
  \usepackage[T1]{fontenc}
  \usepackage[utf8]{inputenc}
  \usepackage{textcomp} % provide euro and other symbols
\else % if luatex or xetex
  \usepackage{unicode-math}
  \defaultfontfeatures{Scale=MatchLowercase}
  \defaultfontfeatures[\rmfamily]{Ligatures=TeX,Scale=1}
\fi
\usepackage{lmodern}
\ifPDFTeX\else  
    % xetex/luatex font selection
\fi
% Use upquote if available, for straight quotes in verbatim environments
\IfFileExists{upquote.sty}{\usepackage{upquote}}{}
\IfFileExists{microtype.sty}{% use microtype if available
  \usepackage[]{microtype}
  \UseMicrotypeSet[protrusion]{basicmath} % disable protrusion for tt fonts
}{}
\makeatletter
\@ifundefined{KOMAClassName}{% if non-KOMA class
  \IfFileExists{parskip.sty}{%
    \usepackage{parskip}
  }{% else
    \setlength{\parindent}{0pt}
    \setlength{\parskip}{6pt plus 2pt minus 1pt}}
}{% if KOMA class
  \KOMAoptions{parskip=half}}
\makeatother
\usepackage{xcolor}
\usepackage[top=0.85in,left=2.75in,footskip=0.75in]{geometry}
\setlength{\emergencystretch}{3em} % prevent overfull lines
\setcounter{secnumdepth}{-\maxdimen} % remove section numbering


\providecommand{\tightlist}{%
  \setlength{\itemsep}{0pt}\setlength{\parskip}{0pt}}\usepackage{longtable,booktabs,array}
\usepackage{calc} % for calculating minipage widths
% Correct order of tables after \paragraph or \subparagraph
\usepackage{etoolbox}
\makeatletter
\patchcmd\longtable{\par}{\if@noskipsec\mbox{}\fi\par}{}{}
\makeatother
% Allow footnotes in longtable head/foot
\IfFileExists{footnotehyper.sty}{\usepackage{footnotehyper}}{\usepackage{footnote}}
\makesavenoteenv{longtable}
\usepackage{graphicx}
\makeatletter
\def\maxwidth{\ifdim\Gin@nat@width>\linewidth\linewidth\else\Gin@nat@width\fi}
\def\maxheight{\ifdim\Gin@nat@height>\textheight\textheight\else\Gin@nat@height\fi}
\makeatother
% Scale images if necessary, so that they will not overflow the page
% margins by default, and it is still possible to overwrite the defaults
% using explicit options in \includegraphics[width, height, ...]{}
\setkeys{Gin}{width=\maxwidth,height=\maxheight,keepaspectratio}
% Set default figure placement to htbp
\makeatletter
\def\fps@figure{htbp}
\makeatother

% Use adjustwidth environment to exceed column width (see example table in text)
\usepackage{changepage}

% marvosym package for additional characters
\usepackage{marvosym}

% cite package, to clean up citations in the main text. Do not remove.
% Using natbib instead
% \usepackage{cite}

% Use nameref to cite supporting information files (see Supporting Information section for more info)
\usepackage{nameref,hyperref}

% line numbers
\usepackage[right]{lineno}

% ligatures disabled
\usepackage{microtype}
\DisableLigatures[f]{encoding = *, family = * }

% create "+" rule type for thick vertical lines
\newcolumntype{+}{!{\vrule width 2pt}}

% create \thickcline for thick horizontal lines of variable length
\newlength\savedwidth
\newcommand\thickcline[1]{%
  \noalign{\global\savedwidth\arrayrulewidth\global\arrayrulewidth 2pt}%
  \cline{#1}%
  \noalign{\vskip\arrayrulewidth}%
  \noalign{\global\arrayrulewidth\savedwidth}%
}

% \thickhline command for thick horizontal lines that span the table
\newcommand\thickhline{\noalign{\global\savedwidth\arrayrulewidth\global\arrayrulewidth 2pt}%
\hline
\noalign{\global\arrayrulewidth\savedwidth}}

% Text layout
\raggedright
\setlength{\parindent}{0.5cm}
\textwidth 5.25in 
\textheight 8.75in

% Bold the 'Figure #' in the caption and separate it from the title/caption with a period
% Captions will be left justified
\usepackage[aboveskip=1pt,labelfont=bf,labelsep=period,justification=raggedright,singlelinecheck=off]{caption}
\renewcommand{\figurename}{Fig}

% Remove brackets from numbering in List of References
\makeatletter
\renewcommand{\@biblabel}[1]{\quad#1.}
\makeatother

% Header and Footer with logo
\usepackage{lastpage,fancyhdr}
\usepackage{epstopdf}
%\pagestyle{myheadings}
\pagestyle{fancy}
\fancyhf{}
%\setlength{\headheight}{27.023pt}
%\lhead{\includegraphics[width=2.0in]{PLOS-submission.eps}}
\rfoot{\thepage/\pageref{LastPage}}
\renewcommand{\headrulewidth}{0pt}
\renewcommand{\footrule}{\hrule height 2pt \vspace{2mm}}
\fancyheadoffset[L]{2.25in}
\fancyfootoffset[L]{2.25in}
\lfoot{\today}
\usepackage{booktabs}
\usepackage{longtable}
\usepackage{array}
\usepackage{multirow}
\usepackage{wrapfig}
\usepackage{float}
\usepackage{colortbl}
\usepackage{pdflscape}
\usepackage{tabu}
\usepackage{threeparttable}
\usepackage{threeparttablex}
\usepackage[normalem]{ulem}
\usepackage{makecell}
\usepackage{xcolor}
\usepackage{caption}
\usepackage{anyfontsize}
\usepackage{dcolumn}
\makeatletter
\@ifpackageloaded{caption}{}{\usepackage{caption}}
\AtBeginDocument{%
\ifdefined\contentsname
  \renewcommand*\contentsname{Table of contents}
\else
  \newcommand\contentsname{Table of contents}
\fi
\ifdefined\listfigurename
  \renewcommand*\listfigurename{List of Figures}
\else
  \newcommand\listfigurename{List of Figures}
\fi
\ifdefined\listtablename
  \renewcommand*\listtablename{List of Tables}
\else
  \newcommand\listtablename{List of Tables}
\fi
\ifdefined\figurename
  \renewcommand*\figurename{Figure}
\else
  \newcommand\figurename{Figure}
\fi
\ifdefined\tablename
  \renewcommand*\tablename{Table}
\else
  \newcommand\tablename{Table}
\fi
}
\@ifpackageloaded{float}{}{\usepackage{float}}
\floatstyle{ruled}
\@ifundefined{c@chapter}{\newfloat{codelisting}{h}{lop}}{\newfloat{codelisting}{h}{lop}[chapter]}
\floatname{codelisting}{Listing}
\newcommand*\listoflistings{\listof{codelisting}{List of Listings}}
\makeatother
\makeatletter
\makeatother
\makeatletter
\@ifpackageloaded{caption}{}{\usepackage{caption}}
\@ifpackageloaded{subcaption}{}{\usepackage{subcaption}}
\makeatother
\ifLuaTeX
  \usepackage{selnolig}  % disable illegal ligatures
\fi
\usepackage[numbers,square,comma]{natbib}
\bibliographystyle{plos2015}
\usepackage{bookmark}

\IfFileExists{xurl.sty}{\usepackage{xurl}}{} % add URL line breaks if available
\urlstyle{same} % disable monospaced font for URLs
\hypersetup{
  pdftitle={Inequality in measuring scholarly success: Variation in the h-index within and between disciplines},
  pdfauthor={Ryan Light; Aaron Gullickson; Jill Ann Harrison},
  pdfkeywords={Metrics, h-index, Gender Inequality, Team Science, Higher
Education},
  hidelinks,
  pdfcreator={LaTeX via pandoc}}



\begin{document}
\vspace*{0.2in}

% Title must be 250 characters or less.
\begin{flushleft}
{\Large
\textbf\newline{Inequality in measuring scholarly success: Variation in
the \emph{h}-index within and between
disciplines} % Please use "sentence case" for title and headings (capitalize only the first word in a title (or heading), the first word in a subtitle (or subheading), and any proper nouns).
}
\newline
\\
% Insert author names, affiliations and corresponding author email (do not include titles, positions, or degrees).
Ryan Light\textsuperscript{1*}, Aaron
Gullickson\textsuperscript{2}, Jill Ann Harrison\textsuperscript{3}
\\
\bigskip
\textbf{1} University of Oregon, Sociology, 632
PLC, Eugene, 97405, OR, USA, \\ \textbf{2} University of
Oregon, Sociology, 727
PLC, Eugene, 97405, OR, USA, \\ \textbf{3} University of
Oregon, Sociology, 604 PLC, Eugene, 97405, OR, USA, 
\bigskip

% Insert additional author notes using the symbols described below. Insert symbol callouts after author names as necessary.
% 
% Remove or comment out the author notes below if they aren't used.
%
% Primary Equal Contribution Note
%\Yinyang These authors contributed equally to this work.

% Additional Equal Contribution Note
% Also use this double-dagger symbol for special authorship notes, such as senior authorship.
%\ddag These authors also contributed equally to this work.

% Current address notes
%\textcurrency Current Address: Dept/Program/Center, Institution Name, City, State, Country % change symbol to "\textcurrency a" if more than one current address note
% \textcurrency b Insert second current address 
% \textcurrency c Insert third current address

% Deceased author note
%\dag Deceased

% Group/Consortium Author Note
%x\textpilcrow Membership list can be found in the Acknowledgments section.

% Use the asterisk to denote corresponding authorship and provide email address in note below.
* light@uoregon.edu

\end{flushleft}

\newpage

{\Large
\textbf\newline{Inequality in measuring scholarly success: Variation in
the \emph{h}-index within and between
disciplines} % Please use "sentence case" for title and headings (capitalize only the first word in a title (or heading), the first word in a subtitle (or subheading), and any proper nouns).
}

\section*{Abstract}
Scholars and university administrators have a vested interest in
building equitable valuation systems of academic work for both practical
(e.g., resource distribution) and more lofty purposes (e.g., what
constitutes ``good'' research). Well-established inequalities in science
pose a difficult challenge to those interested in constructing a
parsimonious and fair method for valuation as stratification occurs
within academic disciplines, but also between them. The \emph{h}-index,
a popular research metric, has been formally used as one such method of
valuation. In this article, we use the case of the \emph{h}-index to
examine how the distribution of research metrics reveal within and
between discipline inequalities. Using bibliometric data from 1960-2019
on over 50,000 high performing scientists - the top 2\% most frequently
cited authors - across 174 disciplines, we construct random effects
within-between models predicting the \emph{h}-index. Results suggest
significant within-discipline variation in several forms, specifically
sole-authorship and female penalties. Results also show that a sole
authorship penalty plays a significant role in well-known
between-discipline variation. Field-specific models emphasize the
``apples-to-oranges,'' or incommensurable, property of cross-discipline
comparison with significant heterogeneity in sole-authorship and female
penalties within fields. Conclusions include continued caution when
using the \emph{h}-index or similar metrics for valuation purposes.


\linenumbers
\section{Introduction}\label{introduction}

From teaching to research, systematic evaluation of academic work
presents a unique set of challenges as the academic disciplines that
broadly organize scholarly labor differ by several relevant factors.
Inequalities experienced by minoritized and women scholars point to
processes of social closure or exclusionary practices that limit the
opportunities for some scholars and not others across disciplines
\citep{hofstra_diversity_2020, lariviere_bibliometrics_2013, xie_sex_1998}.
Cultural and economic differences between disciplines, like the
prevalence of team science or even the definition of success, can affect
the quality and quantity of publication
\citep{fortunato_science_2018, gardner_conceptualizing_2009, stephan_how_2012}.
Scholars have invented dozens of metrics that are formally or informally
used in the evaluation of scholarly research; however, the risk of
apples-to-oranges comparisons given within discipline and between
discipline differences requires greater scrutiny
\citep{hicks_bibliometrics_2015, ioannidis_standardized_2019} alongside
continued efforts towards metrics literacy, or the ability ``critically
assess and effectively and ethically use scholarly
metrics''\citep{maggio_becoming_2022}. Prior research identifies
substantial variation in scholars' understanding of metrics
\citep{hammarfelt_conflicting_2018, kamrani_researchers_2021, rousseau_becoming_2018}
as well as variation in materials for metrics literacy
\citep{maggio_becoming_2022}. The most widely used metric of this type
is the \emph{h}-index, a simple score where a scholar with 10 published
articles with at least 10 citations receives an \emph{h}-index of 10.
The limitations of the \emph{h}-index are well understood, but less is
known about how processes of inequality are embedded both within and
between disciplines. A greater understanding of these processes can help
inform both formal and informal evaluation processes.

From well-known pipeline effects to publication differences to varying
effects of parenthood, research has shown the multifaceted ways some
scholars are obstructed from paths to success at work
\citep{fox_gender_2005, fox_gender_2017, grant_revisiting_2000, long_scientific_1995, morgan_unequal_2021}.
Differences exist in the distribution of resources to disciplines, such
as the prioritization of the natural sciences and engineering in federal
funding and the unequal distribution of resources, like salaries, to
departments, the proxies for disciplines on university campuses
\citep{katz_metrics_2020}. At the same time, numerous scholars have
drawn critical attention to the increasing push to quantify aspects of
social life \citep{mennicken_what_2019}. In the case of scholarly
metrics, critics have raised concerns about how these metrics transform
scholarship into a capitalist-like market (e.g., neoliberalization),
build unproductive individual constraints into academic labor - such as
the anxiety connected to hyper-competitiveness - and ultimately result
in less productive and innovative scholarship
\citep{edwards_academic_2017, muller_tyranny_2019} and efforts to ``game
metrics'' at either the journal or individual-level
\citep{siler_who_2022}. Examining within and between discipline
variation in the \emph{h}-index offers insight into how metrics relate
to inequalities and contributes to research on improving research
metrics for evaluation purposes as well as research on the limitations
of simplifying summaries of academic labor through quantification. Here,
we ask three related questions:

\begin{enumerate}
\def\labelenumi{\arabic{enumi}.}
\tightlist
\item
  What is the variation within and between disciplines in the
  \emph{h}-index?
\item
  How do social factors, specifically gender and sole authorship,
  contribute to this variation?
\item
  How do within and between discipline factors vary across fields (e.g.,
  social sciences, medical sciences, natural sciences)?
\end{enumerate}

While prior work examines the \emph{h}-index by field, discipline, or
gender, less is known about how gender and sole authorship contribute to
\emph{both} within \emph{and} between discipline differences in
\emph{h}-index scores \citep{bihari_review_2023}. This study extends
prior work on metrics in science generally and the \emph{h}-index
specifically by conceptualizing the inequality process as occurring via
intra-disciplinary and inter-disciplinary dynamics. Variation in the
\emph{h}-index may result from ``apples-to-oranges'' comparisons across
disciplines and fields due to specific factors, like gender and sole
authorship, but also occur within disciplines, as scholars experience
differing publishing trajectories dependent on gender and sole
authorship.

\subsection{Inequality in Science}\label{inequality-in-science}

Inequality in science, and in academia generally, persists within and
between disciplines and along well-known axes. Gender is one of the most
studied axes of inequality in science. Research on the scientific
pipeline, for example, shows how obstacles, like cultural stereotypes
about skills differences in math and science and related social
psychological factors such as a sense of belonging, limit pathways to
particular scientific fields for girls
\citep{cech_professional_2011, ma_math_2021, penner_men_2019}. These
pipeline factors continue throughout the research life course as men and
women become segregated in doctoral programs by field and prestige
\citep{weeden_degrees_2017a}. When these significant obstacles are
overcome, women continue to experience inequality in academic work,
including in publication. Publication is both the outcome of academic
labor and the currency of academic careers. How many and the type of
publications produced by a scientist often translates into tangible
resources, like salary raises and the job security of tenure, and less
tangible resources, like prestige. While significant gains have been
made by women in academic work - numerous fields that were male
dominated in the mid- to late-twentieth century are now majority female
- publication remains a potential site of stratification in terms of
both the number of publications scholars accrue over their careers and
the quality of those publications
\citep{xie_sex_1998, light_gender_2009}. Early signs indicate that these
forms of gender inequality may have increased due to the COVID-19
pandemic or a ``pandemic penalty'' \citep{king_pandemic_2021}.

Sociologists of science have spent decades trying to disentangle the
factors associated with gender differences in publication, especially
related to differences in the number of publications and citations.
Nearly 40 years ago, Cole and Zuckerman referred to the ongoing male
advantage in publication and citation counts as the ``productivity
puzzle'' because the causes of this advantage remained difficult to
pinpoint \citep{cole_productivity_1984}. They conclude, ``{[}S{]}ince
gender differences in published productivity persist, the productivity
puzzle has yet to be solved'' (pg. 250). Research suggests that the
productivity puzzle remains. For example, Erin Leahey's research shows
how the level of research specialization interacts with gender to affect
productivity, with consequences for earnings
\citep{leahey_gender_2006, leahey_not_2007}. More recent work provides
evidence that the productivity puzzle in STEM fields results from
variation between men and women in career length and exit rates as
productivity appears to be more equal across shorter time horizons
\citep{huang_historical_2020}. Scholars have also turned to more complex
mechanisms that may help perpetuate gender and racial hierarchies in
academic work. For example, using data on US doctoral recipients,
Hofstra et. al.~find that gender and racial minorities are more likely
to generate innovative scientific work, but that this work is less
likely to be adopted by future researchers with consequences for
academic hiring \citep{hofstra_diversity_2020}.

A prestige puzzle may also coincide with the productivity puzzle as
women may be less likely to publish in the most prestigious journals in
their fields \citep{light_gender_2013}. As top journals have higher
impact factors, the prestige puzzle could have a significant impact on
differences in publication metrics and career outcomes. This form of
inequality may occur for a variety of reasons including a lack of
mentorship, different family and work-based responsibilities between men
and women, and differences in specialization similar to Leahey's work on
specialization and productivity
\citep{leahey_gender_2006, leahey_not_2007, light_gender_2009}. Drawing
on literature on occupational segregation and identity, this work on
elite publication shows how the prestige puzzle has changed over time
within sociology \citep{light_gender_2009}. Earlier cohorts of women
sociologists were significantly less likely to publish in top sociology
journals compared to men regardless of specialty areas. However, as more
women entered sociology, occupational segregation occurred with
subfields becoming sharply gender imbalanced. While baseline models of
the prestige puzzle for more recent cohorts reveal the persistence of
this form of inequality, more complete models that control for
specialization, or occupational identity, show that the contemporary
effect likely operates through these segregation processes. More recent
work on sociology, economics, and political science shows a null effect
of gender on citation when social scientists are situated in similar
disciplinary and sub-field spaces, suggesting that teasing apart the
contexts when a gender penalty persists and when it does not remains an
important concern for those interested in inequality
\citep{lynn_rare_2019}.

Collaboration also likely plays a role with significant historical
differences in coauthorship networks based on gender
\citep{moody_structure_2004}. Recent research on computer scientists by
Jadidi and coauthors finds significant differences in collaboration
between men and women. Men are more likely to have larger coauthorship
networks and to play brokerage roles within them, while women's networks
exhibit increasing gender homophily \citep{jadidi_gender_2018}. These
differences relate to publication outcomes as the network factors
positively affect measures of productivity, including the
\emph{h}-index. Jadidi and coauthors conclude that women ``on average
are less likely to adapt to the collaboration patterns that are related
with success. However, those women who become successful computer
scientists exhibit the same collaborative behavior as their successful
male colleagues'' (pg. 19). Research in both political science and
sociology also identifies how team science affects gender and
publication in these disciplines pointing to how the structure of
scientific work can negatively impact women social scientists
\citep{akbaritabar_gender_2021, teele_gender_2017}.

Collaboration may impact individual-level inequalities beyond gender.
Collaboration is broadly understood as a key aspect of epistemic
culture. Epistemic culture consists of ``those amalgams of arrangements
and mechanisms\ldots which, in a given field, make up how we know what
we know'' (\citep{cetina_epistemic_1999}, pg. 1). The questions that
scholars ask and the strategies that they use to answer them differ by
field with consequences for individuals embedded in specific cultures.
Plainly, the structure of team science matters for publication outcomes.
This suggests that within discipline differences may affect publication
metrics, like the \emph{h}-index, but also points to how inequality may
occur between disciplines as pronounced differences exist between
disciplines in terms of factors like gender composition and team versus
sole authorship.

\subsection{Inequality Between Disciplines and
Fields}\label{inequality-between-disciplines-and-fields}

Disciplinary differences affect inequality along several dimensions. For
example, differences in gender composition may have direct and indirect
effects on how resources are distributed in universities. Disciplinary
cultures also differ and these differences may affect inequality. For
example, disciplines differ in terms of how work is evaluated
\citep{lamont_how_2009} or even how emotions are expressed at work
\citep{koppman_joy_2015}. Moreover, disciplines differ in terms of how
academic work is conducted \citep{huang_historical_2020}. Do scholars
collaborate in teams or are they more likely to work alone? Less is
known about how these between discipline-based inequalities differ from
within discipline inequality. Research on collaboration and citation
impact using the \emph{h}-index shows disciplinary differences in the
effect that collaboration has on impact with more collaboration having a
stronger positive effect in physics and medicine, while having a smaller
effect in the brain sciences or computer science
\citep{parish_dynamics_2018}. Disciplinary differences occur regarding
more tangible resources, like federal funding. Research describes the
substantial differences between fields in terms of federal funding with
implications - a ``domino effect'' - for future non-federal funding and
a potential site of cumulative advantage stratifying disciplines
\citep{lanahan_domino_2016, lynn_15_2021}.

One of the ways that disciplines differ and also one of the ways that
disciplines may be valued differently is how they perform on commonly
used metrics. Time-worn debates in the philosophy of science have tried
to identify the implications of and/or reconcile the so-called ``two
cultures'' or the division between the arts and the sciences
\citep{snow_two_2012}. While this debated formulation may exaggerate
differences, the two cultures perspective draws attention to the
variation that occurs across fields both at a philosophical-level, but
also as a more practical concern. In terms of the latter, prior work has
suggested field normalizing the \emph{h}-index to account for between
field variation \citep{bi_four_2023}. Questions remain about the extent
of variation within and between fields in addition to the variation
occurring within and between disciplines situated in fields. Do
field-level variations in gender composition and team science result in
apple-to-oranges comparisons when using common scholarly metrics like
the \emph{h}-index?

\subsection{The Risks of
Quantification}\label{the-risks-of-quantification}

Quantification has become a central feature of contemporary life as
``{[}a{]}dministration, management, and even mundane daily activities
are increasingly structured around performance measures, cost-benefit
analysis, risk calculations, ratings, and rankings''
\citep[p.~224]{mennicken_what_2019}. Critiques often focus on the risks
of quantification as a central factor in determining worth, which occurs
in a variety of fields from the law to business to education
\citep[p.~4]{muller_tyranny_2019}. This ``metrics fixation'' is part of
the broader process of neoliberalization of education. Neoliberalization
succinctly captures an effort to ``economize everything''
\citep[p.~171]{berg_producing_2016}, such that neoliberal reason becomes
common sense or simply the default rationale people use to make
decisions. Metrics reinforce the notion that individual performance at
work can be easily calculated and compared; therefore, material rewards
like promotions and raises can be fairly and transparently applied. Of
course, metrics often hide as much as they reveal as they simplify a
process by carving away essential components. In universities, the
metrics that help reinforce and are reinforced by neoliberal reason
summarize entire careers for administrators who may have little
understanding of the research that the metrics summarize.

University administrators' use of metrics to evaluate faculty output is
a fairly recent phenomenon. Prior to the development of bibliometric
indicators, evaluation of scholarly research was performed primarily by
disciplinary specialists who offered qualitative assessment of a
research record. While peer assessment is still a central part of
evaluation processes, metrics, such as citation counts, journal impact
factors, and the \emph{h}-index, are now commonly incorporated into
hiring and promotion decisions \citep{mckiernan_metaresearch_2019} and
are seen as more important to untenured scholars than tenured ones
\citep{niles_why_2020}. While many administrators likely have a broad
understanding of such research metrics, such understanding may be
informed by an administrator's own disciplinary and professional
background. Therefore, understanding how metrics relate to within and
between differences across disciplines and fields requires greater
scrutiny. Better understanding of such differences can help
administrators avoid potential pitfalls in evaluation.

Critics have raised concerns about how metrics transform scholarship
into a capitalist-like market at the core of neoliberalization resulting
in ``perverse incentives'' for researchers to publish shoddy or
fraudulent work \citep{edwards_academic_2017}, while simultaneously
resulting in mental health trauma for academic workers experiencing
hypercompetitive markets and suspicious management
\citep{forrester_mental_2021}. From this perspective, administrators'
continued reliance upon metrics serves as a modern form of academic
Taylorism, a production principle developed in the late nineteenth and
early twentieth century that pursued technological solutions to the
``problem'' of worker-related inefficiencies on the shop floor with
little regard to employee satisfaction or wellness
\citep{braverman_labor_1998}. By using technology to set the nature and
pace of production, owners and managers gain greater control over the
labor process itself. To critics, prioritizing metrics creates a demand
for quantity over quality, and by following these demands academic
laborers risk surrendering some degree of control over their own labor
processes.

\subsubsection{The Case of the h-index}\label{the-case-of-the-h-index}

One key metric used for evaluation purposes is the \emph{h}-index or
Hirsch Index. Physicist Jorge Hirsch proposed the \emph{h}-index as a
``useful index to characterize the scientific output of a researcher''
in a 2005 article in the Proceedings of the National Academy of Sciences
\citep{hirsch_index_2005}. While acknowledging the ``potentially
distasteful'' use of metrics for evaluation, he presents quantification
as an economical means of evaluating impact. In this highly cited
article, Hirsch defines the \emph{h}-index as follows: ``A scientist has
index \(h\) if \(h\) of his or her \(N_p\) papers have at least \(h\)
citations each and the other \((N_p-h)\) papers have \(\le h\) citations
each'' (p.~16569). In other words, a scholar with 10 of their 100
publications with a citation count of 10 or higher will have an
\emph{h}-index of 10. He goes on to specify - again with some
acknowledgement that metrics offer a ``rough approximation'' of a
research portfolio - how and when the index could be put to use: ``Based
on typical \emph{h} and \emph{m} values found, I suggest (with large
error bars) that for faculty at major research universities,
\(h \approx 12\) might be a typical value for advancement to tenure
(associate professor) and that \(h \approx 18\) might be a typical value
for advancement to full professor'' (p.~16571). In sum, this publication
announced a simple means of evaluating research impact and permission to
use the metric for evaluation purposes.

The immediate response to the \emph{h}-index was largely positive with
features in top scientific journals; however, some criticism of the
index also quickly appeared \citep{barnes_hindex_2017}. Critics
identified a range of issues from the relationship between the
\emph{h}-index and career length as well as the effect of self-citation
\citep[ among others]{kelly_index_2006, purvis_index_2006}. However, the
\emph{h}-index and variants have proven enormously popular both in the
bibliometrics and science of science communities and among university
administrators seeking quick and cheap ways to evaluate scholars,
including universities and science funding agencies
\citep{barnes_hindex_2017}. The \emph{h}-index is included as a key
quantitative metric for annual review and/or tenure and promotion in
faculty handbooks in a range of departments and schools in the United
States (c.f., handbooks from the Boston University School of Public
Health \citep{bostonuniversity_boston_2018}, the Ohio State University
Department of Surgery \citep{ohiostateuniversity_department_2014} or
Oregon State University's College of Business
\citep{oregonstateuniversity_oregon_2020}). Survey research in Germany
on whether and how scholars understand the importance of the
\emph{h}-index indicates that natural scientists widely understand the
importance of the \emph{h}-index to their careers, but scholars in the
humanities and social sciences do not \citep{kamrani_researchers_2021}.
This variation in knowledge motivates calls for expanding metrics
literacy efforts
\citep{ioannidis_citation_2020, maggio_becoming_2022, rousseau_becoming_2018}.
In sum, despite some criticism, the \emph{h}-index has been widely
adopted although perhaps not widely understood. This analysis
contributes to the broader literature on metrics and science by
examining the factors contributing to within and between discipline
variation in the \emph{h}-index.

\subsection{Hypotheses}\label{hypotheses}

We develop the following hypotheses to better understand within and
between discipline differences in the \emph{h}-index, or Hirsch Index,
based on the prior literature. Consistent with work on gender inequality
and science and particularly work on the productivity and prestige
puzzles, we examine the following:

\subsubsection{Within Discipline
Hypotheses}\label{within-discipline-hypotheses}

In light of research on team science and its impact on academic careers,
we examine the following:

\begin{quote}
\emph{Female Penalty Hypothesis (H1):} Authors with names more common
among women will have lower \emph{h}-index scores, on average.
\end{quote}

\begin{quote}
\emph{Sole Author Hypothesis (H2):} Authors with a higher share of
sole-authored publications will have lower \emph{h}-index scores, on
average.
\end{quote}

\subsubsection{Between Disciplines
Hypotheses}\label{between-disciplines-hypotheses}

Although less well understood, based on the research describing
disciplinary differences in culture, such as propensity to collaborate,
and material differences in resources, we examine two related
hypotheses:

\begin{quote}
\emph{Feminized Discipline Hypothesis (H3a):} Disciplines with a higher
share of women will have lower \emph{h}-index scores, on average.
\end{quote}

\begin{quote}
\emph{Teamwork Variation Hypothesis (H3b):} Disciplines with a higher
share of sole authorship will have lower \emph{h}-index scores, on
average.
\end{quote}

\subsubsection{Field-level Hypothesis}\label{field-level-hypothesis}

Finally, a field-level view may help disentangle cultural and
compositional effects and shed light on the question of
``apples-to-oranges'' comparison within and between fields. We,
therefore, evaluate the following hypothesis:

\begin{quote}
\emph{Field Variation Hypothesis (H4):} Significant field differences
will exist in the relationship of the \emph{h}-index to gender and sole
authorship at both the individual and disciplinary level.
\end{quote}

\section{Materials and Methods}\label{materials-and-methods}

To evaluate these hypotheses, we analyze data on high performing
scholars according to well-known metrics. Ioannidis and coauthors
\citep{ioannidis_standardized_2019} collected author-level bibliometrics
on 100,000 high performing scholars from the Scopus database. They
updated this data through 2019 and expanded to include the top 2\% of
authors overall across a wide range of disciplines
\citep{ioannidis_updated_2020}. These data suffer from several
limitations, including coverage differences between disciplines within
the Scopus database \citep{mongeon_journal_2016, singh_journal_2021}.
Nonetheless, these well-curated data represent a unique opportunity to
evaluate within and between disciplinary differences in \emph{h}-index
scores. We also see these data as offering a conservative test of such
differences as variation is likely to widen when moving beyond scholars
who are in the top 2\% on these metrics.

We reduce the data along several dimensions to address some of the
limitations of the Scopus database and for analytic purposes. First, we
limit the analysis only to the 68,016 scholars in the United States to
account for geographic variation in the Scopus database and geographic
variation in how universities are structured. We then further restrict
the analytical data to those who first published in 1960 or after and
those who last published in 2017 or later to identify active scholars
and reduce the impact of historical, rather than contemporary, patterns.
These reductions, along with a small number of missing values on the
variables below, leave us a final analytical sample size of 54,825
scholars nested in 174 disciplines.

The dependent variable for the analysis is the \emph{h}-index. The key
independent variables are the probability of female name and the
percentage of sole authored publications. We also include additional
control variables of the count of scholars from the same university in
the same dataset to measure highly productive university environments, a
specialization score measured as the proportion of total articles
appearing in the main discipline for each author, and career length,
measured by the years between each scholar's first and most recent
publication.

Estimating gender is problematic for numerous reasons including the
typical reliance on government-provided data that often assumes and
contributes to the gender binary \citep{mihaljevic_reflections_2019},
and these methods should be used with caution and only when necessary.
In this case, names are our only means of estimating gender. Here, we
draw on first name data from the 1940-1990 US Social Security
Administration to assign a probability of female name using the gender
package in R \citep{blevins_jane_2015}. We use this probability directly
in our models, rather than assigning an arbitrary cutoff for a binary
gender assignment. A substantial number of cases (18\%) are missing on
this variable, usually because they were identified by initials rather
than a full first name. Rather than lose this many cases, we use
multiple imputation to assign values to the probability of female name
based on the respondent's other variables, including discipline. We
impute five complete datasets and pool analyses across these datasets
for all of the models presented here.

We also consider how the patterns we observe may differ within large
fields among disciplines. To explore this issue, we divide the total 174
disciplines into five large fields of the humanities, medical (including
public health), professional, social sciences (including policy), and
science, technology, engineering, and math (STEM). These fields differ
from the fields provided in the original data but more closely
correspond to the division of disciplines within the American academy.
The supplementary materials show a full list of disciplines, which
disciplines were assigned to each field, and summary statistics for each
discipline.

Table~\ref{tbl-desc} presents the descriptive statistics for all
variables used in the analyses for the total sample and across the five
different fields.

\begin{table}
\begin{adjustwidth}{-1.5in}{0in} % Comment out/remove adjustwidth environment if table fits in text column.
\caption{\label{tbl-desc}Mean and standard deviation on key variables
for the whole sample and by field, based on first complete dataset}

\centering{

\fontsize{9.0pt}{10.8pt}\selectfont
\begin{tabular*}{\linewidth}{@{\extracolsep{\fill}}lrrrrrr}
\toprule
 &  & \multicolumn{5}{c}{Fields} \\ 
\cmidrule(lr){3-7}
Variable & All & Humanities & Medical & Prof. & Soc. Sci. & STEM \\ 
\midrule\addlinespace[2.5pt]
h-index & 44.1 (20.9) & 19.8 (10.4) & 52.2 (21.7) & 29.4 (10.7) & 33.1 (15.0) & 38.8 (17.6) \\ 
\% female & 19.0 (37.6) & 30.6 (45.0) & 22.0 (40.0) & 21.4 (39.5) & 27.8 (43.6) & 13.5 (32.0) \\ 
\% sole author & 10.7 (14.4) & 56.0 (31.4) & 8.1 (9.4) & 18.2 (16.8) & 24.6 (22.9) & 9.1 (11.5) \\ 
career length & 35.3 (10.1) & 33.6 (11.2) & 36.2 (9.3) & 32.2 (9.9) & 34.7 (10.6) & 34.7 (10.6) \\ 
specialization & 0.73 (0.19) & 0.64 (0.21) & 0.76 (0.18) & 0.75 (0.21) & 0.72 (0.19) & 0.70 (0.20) \\ 
uni. count & 198 (202) & 255 (204) & 185 (201) & 166 (203) & 246 (209) & 204 (200) \\ 
N & 54,825 & 856 & 25,581 & 1,228 & 3,987 & 23,173 \\ 
\bottomrule
\end{tabular*}
\begin{minipage}{\linewidth}
Note: standard deviations shown in parenthesis\\
\end{minipage}

}
\end{adjustwidth}
\end{table}%

\subsection{Analytic Strategy}\label{analytic-strategy}

We model variation in \emph{h}-index scores across scholars using random
effects within-between (REWB) models \citep{bell_fixed_2019}. Random
effects within-between (REWB) models are a variant of multilevel models
that allows researchers to estimate the effect of a given predictor
variable both within (as per a standard ``fixed effects'' model) and
between the higher level clusters (in this case disciplines). In
general, the structure of the REWB model is:

\[y_{ij}=\beta_{0j}+\beta_1(x_{ij}-\bar{x}_{.j})+\beta_2(\bar{x}_{.j})+\upsilon_{0j}+\epsilon_{ij}\]

Where \(y_{ij}\) is the outcome for the \(i\)th unit in the \(j\)th
cluster and \(x_{ij}\) is the predictor variable for the \(i\)th unit in
the \(j\)th cluster. \(\upsilon_{0j}\) and \(\epsilon_{ij}\) are
cluster-level and individual-level random errors, respectively. Because
the mean of \(x\) for cluster \(j\) (\(\bar{x}_{.j}\)) is included in
the model and \(x_{ij}\) values are mean centered by cluster, the
\(\beta_1\) parameter is identical to that of a fixed-effects model in
which all between variance is absorbed by cluster-level dummy variables.
However, the REWB model also includes an estimate of the between cluster
effect of \(x\) estimated in \(\beta_2\), which is impossible in a
fixed-effects model. This \(\beta_2\) term is equivalent to the estimate
obtained by aggregating data to the higher level and examining the
relationship between the means of \(x\) and \(y\). Thus, this model
maintains the advantage of traditional fixed effects models by absorbing
all cluster level differences when estimating lower level parameters,
while at the same time allowing for an analysis of the ``contextual''
relationships at the higher level.

We use this feature to estimate both within and between effects of
gender and sole authorship on a scholar's \emph{h}-index. Both the
probability of a female name variable and the sole authorship variable
are mean centered by discipline at the individual level so that the
interpretation of their effect is solely among scholars within the same
discipline. For the between discipline effects, we calculate the mean
probability of a female name and the mean sole authorship of each
discipline. For probability of a female name, we use a 0 to 100
percentage scale at the disciplinary level for ease of interpretation of
coefficients as the expected change in the \emph{h}-index for a one
percentage point increase in the mean probability of a female name. For
sole authorship, we use a natural log transformation of the mean percent
sole-authorship variable at the disciplinary level because exploratory
analysis indicated a negative diminishing returns relationship between
sole-authorship and the \emph{h}-index at this level.

The control variables of career length, specialization, and university
publication count are only included as grand mean centered individual
level variables. We also standardize university publication count to
have a mean of zero and a standard deviation of one for ease of
interpretation.

In addition to the models for the full data, we also run these same
models separately for each of these fields to explore differences in
patterns across fields.

\section{Results}\label{results}

We begin by analyzing a partition of the variance in the \emph{h}-index
within and between disciplines in a null multilevel model. The
percentage of the total variation in the \emph{h}-index that occurs
between disciplines is given by the intraclass correlation coefficient
(ICC) of this model. Table~\ref{tbl-partition} shows the ICC for the
model across all observations as well as separately by field. In total,
roughly a third of the variation in the \emph{h}-index occurs between
disciplines.

\begin{table}
\begin{adjustwidth}{-1.5in}{0in} % Comment out/remove adjustwidth environment if table fits in text column.
\caption{\label{tbl-partition}Partition of variance in \emph{h}-index
scores between and within disciplines}

\centering{

\fontsize{9.0pt}{10.8pt}\selectfont
\begin{tabular*}{\linewidth}{@{\extracolsep{\fill}}lrrrrrr}
\toprule
 &  & \multicolumn{5}{c}{Field} \\ 
\cmidrule(lr){3-7}
Grouping & All (\%) & Humanities (\%) & Medical (\%) & Prof. (\%) & Soc. Sci. (\%) & STEM (\%) \\ 
\midrule\addlinespace[2.5pt]
Between discipline & 32.0 & 53.6 & 20.4 & 16.2 & 31.3 & 20.1 \\ 
Within discipline & 68.0 & 46.4 & 79.6 & 83.8 & 68.7 & 79.9 \\ 
\bottomrule
\end{tabular*}

}
\end{adjustwidth}
\end{table}%

The ICC is substantial across all fields as well, but also varies
substantially. Slightly more than half the variation in the
\emph{h}-index among scholars in the humanities is between disciplines,
while only one-fifth or less of the variation in the \emph{h}-index is
between disciplines among scholars in the medical, STEM, and
professional fields.

Table~\ref{tbl-models-full} presents the multilevel models predicting
the \emph{h}-index across all disciplines. Model 1 predicts
\emph{h}-index scores by the individual and disciplinary variables for
gender. Model 2 predicts \emph{h}-index scores by the individual and
disciplinary variables for sole authorship. Model 3 includes both sets
of predictor variables from Models 1 and 2 together. Finally, Model 4
includes additional control variables for career length, specialization,
and highly productive universities. The estimates for the key predictor
variables of gender and sole-authorship are robust to these additional
controls, although gender differences do decline in size somewhat. We
use the results from Model 4 to describe overall patterns below, except
where noted otherwise.

\begin{table}

\caption{\label{tbl-models-full}Multilevel models predicting
\emph{h}-index score across all disciplines with clustering at the
disciplinary level.}

\centering{

[!t]
\caption{Statistical models}
\begin{center}
\begin{tabular}{l c c c c}
\hline
 & Model 1 & Model 2 & Model 3 & Model 4 \\
\hline
Intercept                               & $39.32^{*}$ & $69.64^{*}$  & $68.86^{*}$  & $68.91^{*}$  \\
                                        & $(1.80)$    & $(2.12)$     & $(2.13)$     & $(2.06)$     \\
Prob. [0-1] of female name$\dagger$     & $-3.00^{*}$ &              & $-3.67^{*}$  & $-2.63^{*}$  \\
                                        & $(0.23)$    &              & $(0.22)$     & $(0.21)$     \\
Disc. mean percent female name          & $-0.15^{*}$ &              & $0.10^{*}$   & $0.11^{*}$   \\
                                        & $(0.07)$    &              & $(0.05)$     & $(0.05)$     \\
Percent sole authored pubs$\dagger$     &             & $-0.41^{*}$  & $-0.42^{*}$  & $-0.47^{*}$  \\
                                        &             & $(0.01)$     & $(0.01)$     & $(0.01)$     \\
Disc. mean percent sole authored (log)  &             & $-12.87^{*}$ & $-13.39^{*}$ & $-13.53^{*}$ \\
                                        &             & $(0.79)$     & $(0.82)$     & $(0.79)$     \\
Career length*                          &             &              &              & $0.35^{*}$   \\
                                        &             &              &              & $(0.01)$     \\
Specialization [0-1]*                   &             &              &              & $-2.66^{*}$  \\
                                        &             &              &              & $(0.44)$     \\
Highly productive uni. count (stdized)* &             &              &              & $1.72^{*}$   \\
                                        &             &              &              & $(0.07)$     \\
\hline
Residual SD between discipline          & $11.86$     & $7.41$       & $7.33$       & $7.05$       \\
Residual SD within discipline           & $17.41$     & $16.83$      & $16.78$      & $16.34$      \\
N (discipline)                          & $174$       & $174$        & $174$        & $174$        \\
N (individual)                          & $54825$     & $54825$      & $54825$      & $54825$      \\
\hline
\multicolumn{5}{l}{\scriptsize{$^{*}p<0.05$. Standard errors shown in parenthesis. $\dagger$=discipline mean centered; *=grand mean centered.}}
\end{tabular}
\label{table:coefficients}
\end{center}

}

\end{table}%

Consistent with the productivity puzzle, female scholars have an
\emph{h}-index approximately 2.63 points lower than male scholars in the
same discipline, on average. More frequent sole authorship is also
associated with a lower \emph{h}-index. Within the same discipline, a
one percentage point increase in the percent of sole-authored
publications for a given scholar is associated with a 0.47 lower
\emph{h}-index score, on average.

Additionally, we find substantial differences between disciplines in
\emph{h}-index scores based on the feminization of the discipline and
sole-authorship norms. Sole authorship behaves as expected. A one
percent increase in the mean percent sole-authored publications in a
discipline is associated with a 0.135 decline in the mean \emph{h}-index
for that discipline. Thus, cultural norms of more sole-authorship within
a discipline contribute to lower overall \emph{h}-index scores for that
discipline.

The feminization of a discipline, as indicated by the mean percent
female name within the discipline, has a more complex relationship to
\emph{h}-index scores. Model 1 shows a slightly negative association
between percent female and mean \emph{h}-index scores across
disciplines. However, this slightly negative association is a spurious
byproduct of the underlying tendency for more feminized disciplines to
also be more focused on sole authorship (\(r=0.30\)). Model 3 shows
that, once we hold constant the disciplinary tendency for
sole-authorship, the association between feminization and \emph{h}-index
scores becomes slightly positive. In the final model, we estimate that a
one percentage point increase in the percent female name within a
discipline is associated with a 0.11 increase in the mean \emph{h}-index
score for that discipline. Thus, when we compare disciplines with
similar sole-authorship tendencies, feminized disciplines actually have
a slight advantage in mean \emph{h}-index scores, even though, within
disciplines, women are still disadvantaged relative to men.

Table~\ref{tbl-models-field} presents models equivalent to Model 4 in
Table~\ref{tbl-models-full}, but separated by field. The most notable
change in these sets of models is that there is no observable effect of
disciplinary feminization within all fields, except for the Social
Sciences. In contrast, we observed a moderate positive effect of
disciplinary feminization when we pool all fields. This finding implies
that the slight positive effect of disciplinary feminization was driven
by compositional issues between fields and differences in overall field
productivity. The remaining variables are consistent in direction, but
vary substantially in magnitude across fields. The within-discipline
differences by gender are smallest in the humanities and STEM fields and
largest in the medical field. Similarly, the within-discipline
differences by sole-authorship are smallest in the humanities and
largest in the medical field. The between-discipline effect of
sole-authorship is largest in the medical field and smallest in the
professional field where it is only a third as large.

\begin{table}
\begin{adjustwidth}{-1.5in}{0in} % Comment out/remove adjustwidth environment if table fits in text column.
\caption{\label{tbl-models-field}Multilevel models predicting
\emph{h}-index score within each field. Clustering at the disciplinary
level.}

\centering{

[!t]
\caption{Statistical models}
\begin{center}
\begin{tabular}{l c c c c c}
\hline
 & Humanities & Medical & Prof. & Soc. Sci. & STEM \\
\hline
Intercept                               & $79.45^{*}$  & $93.04^{*}$  & $47.21^{*}$ & $77.81^{*}$  & $58.73^{*}$  \\
                                        & $(12.68)$    & $(6.37)$     & $(8.23)$    & $(4.57)$     & $(5.18)$     \\
Prob. [0-1] of female name$\dagger$     & $-1.30^{*}$  & $-3.73^{*}$  & $-1.57^{*}$ & $-2.11^{*}$  & $-1.25^{*}$  \\
                                        & $(0.50)$     & $(0.31)$     & $(0.76)$    & $(0.43)$     & $(0.34)$     \\
Disc. mean percent female name          & $-0.02$      & $0.07$       & $-0.01$     & $-0.13^{*}$  & $0.12$       \\
                                        & $(0.09)$     & $(0.08)$     & $(0.21)$    & $(0.05)$     & $(0.17)$     \\
Percent sole authored pubs$\dagger$     & $-0.16^{*}$  & $-0.66^{*}$  & $-0.26^{*}$ & $-0.24^{*}$  & $-0.49^{*}$  \\
                                        & $(0.01)$     & $(0.01)$     & $(0.02)$    & $(0.01)$     & $(0.01)$     \\
Disc. mean percent sole authored (log)  & $-15.38^{*}$ & $-21.79^{*}$ & $-6.84^{*}$ & $-13.66^{*}$ & $-10.57^{*}$ \\
                                        & $(3.08)$     & $(3.02)$     & $(3.34)$    & $(1.19)$     & $(1.83)$     \\
Career length*                          & $0.13^{*}$   & $0.45^{*}$   & $0.29^{*}$  & $0.31^{*}$   & $0.30^{*}$   \\
                                        & $(0.02)$     & $(0.01)$     & $(0.03)$    & $(0.02)$     & $(0.01)$     \\
Specialization [0-1]*                   & $-3.13^{*}$  & $0.03$       & $-0.46$     & $-6.31^{*}$  & $-4.36^{*}$  \\
                                        & $(1.10)$     & $(0.79)$     & $(1.55)$    & $(1.03)$     & $(0.57)$     \\
Highly productive uni. count (stdized)* & $0.55^{*}$   & $1.51^{*}$   & $0.60^{*}$  & $0.38^{*}$   & $2.16^{*}$   \\
                                        & $(0.21)$     & $(0.12)$     & $(0.29)$    & $(0.18)$     & $(0.10)$     \\
\hline
Residual SD between discipline          & $4.51$       & $6.70$       & $4.27$      & $3.13$       & $6.22$       \\
Residual SD within discipline           & $5.98$       & $18.69$      & $9.48$      & $11.31$      & $14.41$      \\
N (discipline)                          & $15$         & $52$         & $12$        & $28$         & $67$         \\
N (individual)                          & $856$        & $25581$      & $1228$      & $3987$       & $23173$      \\
\hline
\multicolumn{6}{l}{\scriptsize{$^{*}p<0.05$. Standard errors shown in parenthesis. $\dagger$=discipline mean centered; *=grand mean centered.}}
\end{tabular}
\label{table:coefficients}
\end{center}

}
\end{adjustwidth}
\end{table}%

\section{Discussion}\label{discussion}

The substantial heterogeneity across disciplines (roughly a third of all
variation) indicates that comparing the \emph{h}-index across
disciplines is problematic because of overall differences in culture,
productivity, and citation patterns across disciplines. Additionally, we
found that fields vary in the level of disciplinary heterogeneity. Some
fields, like the medical, STEM, and professional fields, seem to have
more shared culture and practices in terms of publishing productivity,
while in other fields, we observe substantial disciplinary
heterogeneity. Scholars in fields with greater commonality may
misperceive the comparability of the \emph{h}-index across disciplines
more broadly.

Results indicate that, within disciplines, gender and sole authorship
affect the \emph{h}-index. Consistent with H1 and H2, we find that women
have lower \emph{h}-index scores than men and that more sole-authorship
is associated with lower \emph{h}-index scores, respectively. Thus,
among scholars in the same discipline, variation in both ascriptive
characteristics such as gender and research practices can generate
substantial inequalities in \emph{h}-index scores.

We find evidence of between discipline differences as well. Disciplines
with more overall sole authorship have substantially lower mean
\emph{h}-index scores overall, consistent with the teamwork variation
hypothesis (H3b). Support for the feminized discipline hypothesis (H3a)
is more mixed. In the full models, results vary by whether disciplinary
sole-authorship practice is included in the model because feminized
disciplines are more likely to be high sole-authorship disciplines as
well. In field specific models, we find a negative feminization effect
on \emph{h}-index scores only for the Social Sciences, and on effect for
the remaining disciplines, even when controlling for disciplinary sole
authorship. Thus, to the extent, that the feminization effect exists, it
appears limited to certain fields.

We also observe that both the within discipline and between discipline
effects vary substantially across fields, consistent with the field
variation hypothesis (H4). This heterogeneity in models across fields
implies an additional apples-to-oranges comparison problem when
comparing scholars across fields. Even if some attention is paid to the
overall predictors of productivity within and between disciplines when
evaluating scholars, field heterogeneity in the effects of these
predictors will complicate the ability to make accurate comparisons.

Data limitations suggest several avenues for future research. First, the
data select on high-performing scholars, and, therefore do not
generalize to the broader population of academics. While we believe that
this limitation likely results in a conservative estimate of within and
between discipline inequality, more data on academia writ large are
required to verify this claim. Second, the estimation of gender suffers
from well-known limitations. Data linking self-reported gender beyond
the gender binary to citation data would be a welcome resource. Gender
estimation also required reducing the dataset by country. Third, and
along similar lines, most scholarship on publication and inequality
focuses on gender because it is possible to infer gender from names,
despite some weaknesses in the approach. Other aspects of inequality,
like race and class, remain under-studied. Future research would benefit
from data linking self-reported race and class to large bibliometric
data to better understand the broad effects of inequality in academic
work.

\section{Conclusion}\label{conclusion}

This analysis provides further evidence that metrics in performance
evaluation in academia should be used with caution and efforts towards
metrics literacy should include both within and between discipline
variation. Metrics are subject to within and between discipline biases
that hinder their value in making both intradisciplinary and
interdisciplinary comparisons. Of course, these comparisons are exactly
what the quantification of scholarly work proposes to facilitate. Like
other forms of quantification, academic metrics simplify complex
processes at a cost. This cost can reinforce existing inequalities and
even generate new ``automated inequalities''
\citep{eubanks_automating_2018}. Those using metrics for evaluation, as
Koopman and Galton \citep{koopman_galton_2023} write about data usage
generally, ``need to be fervently attentive to the ways in which
inequalities may be designed into their data'' (pg. 16). In this vein,
this analysis provides evidence of within and between discipline
differences in the \emph{h}-index with a focus on gender and sole
authorship, but variation in scholarly metrics may extend to other
dimensions of power, including race, class, and sexual orientation, or
be related to other academic practices, like research funding, teaching
loads, or service responsibilities.

\section{Acknowledgements}\label{acknowledgements}

We thank Nicholas Theis and jimi adams for their helpful comments on
previous drafts of this article.

\section{References}\label{references}

\renewcommand{\bibsection}{}
\begin{thebibliography}{10}

\bibitem{hofstra_diversity_2020}
Hofstra B, Kulkarni VV, {Munoz-Najar Galvez} S, He B, Jurafsky D, McFarland DA.
\newblock The Diversity--Innovation Paradox in Science.
\newblock Proceedings of the National Academy of Sciences. 2020;117(17):9284--9291.

\bibitem{lariviere_bibliometrics_2013}
Larivi{\`e}re V, Ni C, Gingras Y, Cronin B, Sugimoto CR.
\newblock Bibliometrics: {{Global}} Gender Disparities in Science.
\newblock Nature. 2013;504(7479):211--213.

\bibitem{xie_sex_1998}
Xie Y, Shauman KA.
\newblock Sex {{Differences}} in {{Research Productivity}}: {{New Evidence}} about an {{Old Puzzle}}.
\newblock American Sociological Review. 1998;63(6):847--870.
\newblock doi:{10.2307/2657505}.

\bibitem{fortunato_science_2018}
Fortunato S, Bergstrom CT, B{\"o}rner K, Evans JA, Helbing D, Milojevi{\'c} S, et~al.
\newblock Science of Science.
\newblock Science. 2018;359(6379):eaao0185.
\newblock doi:{10.1126/science.aao0185}.

\bibitem{gardner_conceptualizing_2009}
Gardner SK.
\newblock Conceptualizing {{Success}} in {{Doctoral Education}}: {{Perspectives}} of {{Faculty}} in {{Seven Disciplines}}.
\newblock The Review of Higher Education. 2009;32(3):383--406.
\newblock doi:{10.1353/rhe.0.0075}.

\bibitem{stephan_how_2012}
Stephan P.
\newblock How Economics Shapes Science.
\newblock Harvard University Press; 2012.

\bibitem{hicks_bibliometrics_2015}
Hicks D, Wouters P, Waltman L, {de Rijcke} S, Rafols I.
\newblock Bibliometrics: {{The Leiden Manifesto}} for Research Metrics.
\newblock Nature. 2015;520(7548):429--431.
\newblock doi:{10.1038/520429a}.

\bibitem{ioannidis_standardized_2019}
Ioannidis JPA, Baas J, Klavans R, Boyack KW.
\newblock A Standardized Citation Metrics Author Database Annotated for Scientific Field.
\newblock PLOS Biology. 2019;17(8):e3000384.
\newblock doi:{10.1371/journal.pbio.3000384}.

\bibitem{maggio_becoming_2022}
Maggio LA, Jeffrey A, Haustein S, Samuel A.
\newblock Becoming Metrics Literate: {{An}} Analysis of Brief Videos That Teach about the h-Index.
\newblock PLOS ONE. 2022;17(5):e0268110.
\newblock doi:{10.1371/journal.pone.0268110}.

\bibitem{hammarfelt_conflicting_2018}
Hammarfelt B, Haddow G.
\newblock Conflicting Measures and Values: {{How}} Humanities Scholars in {{Australia}} and {{Sweden}} Use and React to Bibliometric Indicators.
\newblock Journal of the Association for Information Science and Technology. 2018;69(7):924--935.
\newblock doi:{10.1002/asi.24043}.

\bibitem{kamrani_researchers_2021}
Kamrani P, Dorsch I, Stock WG.
\newblock Do Researchers Know What the H-Index Is? {{And}} How Do They Estimate Its Importance?
\newblock Scientometrics. 2021;126(7):5489--5508.
\newblock doi:{10.1007/s11192-021-03968-1}.

\bibitem{rousseau_becoming_2018}
Rousseau R, Egghe L, Guns R.
\newblock Becoming {{Metric-Wise}}: {{A Bibliometric Guide}} for {{Researchers}}.
\newblock Chandos Publishing; 2018.

\bibitem{fox_gender_2005}
Fox MF.
\newblock Gender, Family Characteristics, and Publication Productivity among Scientists.
\newblock Social Studies of Science. 2005;35(1):131--150.

\bibitem{fox_gender_2017}
Fox MF, Whittington K, Linkova M.
\newblock Gender,(in) Equity, and the Scientific Workforce.
\newblock Handbook of science and technology studies. 2017; p. 701--731.

\bibitem{grant_revisiting_2000}
Grant L, Kennelly I, Ward KB.
\newblock Revisiting the Gender, Marriage, and Parenthood Puzzle in Scientific Careers.
\newblock Women's Studies Quarterly. 2000;28(1/2):62--85.

\bibitem{long_scientific_1995}
Long JS, Fox MF.
\newblock Scientific Careers: {{Universalism}} and Particularism.
\newblock Annual review of sociology. 1995;21(1):45--71.

\bibitem{morgan_unequal_2021}
Morgan AC, Way SF, Hoefer MJD, Larremore DB, Galesic M, Clauset A.
\newblock The Unequal Impact of Parenthood in Academia.
\newblock Science Advances. 2021;7(9):eabd1996.
\newblock doi:{10.1126/sciadv.abd1996}.

\bibitem{katz_metrics_2020}
Katz Y, Matter U.
\newblock Metrics of {{Inequality}}: {{The Concentration}} of {{Resources}} in the {{U}}.{{S}}. {{Biomedical Elite}}.
\newblock Science as Culture. 2020;29(4):475--502.
\newblock doi:{10.1080/09505431.2019.1694882}.

\bibitem{mennicken_what_2019}
Mennicken A, Espeland WN.
\newblock What's New with Numbers? {{Sociological}} Approaches to the Study of Quantification.
\newblock Annual Review of Sociology. 2019;45:223--245.

\bibitem{edwards_academic_2017}
Edwards MA, Roy S.
\newblock Academic {{Research}} in the 21st {{Century}}: {{Maintaining Scientific Integrity}} in a {{Climate}} of {{Perverse Incentives}} and {{Hypercompetition}}.
\newblock Environmental Engineering Science. 2017;34(1):51--61.
\newblock doi:{10.1089/ees.2016.0223}.

\bibitem{muller_tyranny_2019}
Muller JZ.
\newblock The Tyranny of Metrics.
\newblock Princeton University Press; 2019.

\bibitem{siler_who_2022}
Siler K, Larivi{\`e}re V.
\newblock Who Games Metrics and Rankings? {{Institutional}} Niches and Journal Impact Factor Inflation.
\newblock Research Policy. 2022;51(10):104608.

\bibitem{bihari_review_2023}
Bihari A, Tripathi S, Deepak A.
\newblock A Review on H-Index and Its Alternative Indices.
\newblock Journal of Information Science. 2023;49(3):624--665.
\newblock doi:{10.1177/01655515211014478}.

\bibitem{cech_professional_2011}
Cech E, Rubineau B, Silbey S, Seron C.
\newblock Professional {{Role Confidence}} and {{Gendered Persistence}} in {{Engineering}}.
\newblock American Sociological Review. 2011;76(5):641--666.
\newblock doi:{10.1177/0003122411420815}.

\bibitem{ma_math_2021}
Ma Y, Xiao S.
\newblock Math and {{Science Identity Change}} and {{Paths}} into and out of {{STEM}}: {{Gender}} and {{Racial Disparities}}.
\newblock Socius. 2021;7:23780231211001978.
\newblock doi:{10.1177/23780231211001978}.

\bibitem{penner_men_2019}
Penner AM, Willer R.
\newblock Men's {{Overpersistence}} and the {{Gender Gap}} in {{Science}} and {{Mathematics}}.
\newblock Socius. 2019;5:2378023118821836.
\newblock doi:{10.1177/2378023118821836}.

\bibitem{weeden_degrees_2017a}
Weeden KA, Th{\'e}baud S, Gelbgiser D.
\newblock Degrees of Difference: {{Gender}} Segregation of {{US}} Doctorates by Field and Program Prestige.
\newblock Sociological Science. 2017;4:123--150.

\bibitem{light_gender_2009}
Light R.
\newblock Gender Stratification and Publication in {{American}} Science: {{Turning}} the Tools of Science Inward.
\newblock Sociology Compass. 2009;3(4):721--733.

\bibitem{king_pandemic_2021}
King MM, Frederickson ME.
\newblock The {{Pandemic Penalty}}: {{The Gendered Effects}} of {{COVID-19}} on {{Scientific Productivity}}.
\newblock Socius. 2021;7:23780231211006977.
\newblock doi:{10.1177/23780231211006977}.

\bibitem{cole_productivity_1984}
Cole J, Zuckerman H.
\newblock The {{Productivity Puzzle}}.
\newblock Advances in Motivation and Achievement. 1984; p. 217--258.

\bibitem{leahey_gender_2006}
Leahey E.
\newblock Gender {{Differences}} in {{Productivity}}: {{Research Specialization}} as a {{Missing Link}}.
\newblock Gender \& Society. 2006;20(6):754--780.
\newblock doi:{10.1177/0891243206293030}.

\bibitem{leahey_not_2007}
Leahey E.
\newblock Not by {{Productivity Alone}}: {{How Visibility}} and {{Specialization Contribute}} to {{Academic Earnings}}.
\newblock American Sociological Review. 2007;72(4):533--561.
\newblock doi:{10.1177/000312240707200403}.

\bibitem{huang_historical_2020}
Huang J, Gates AJ, Sinatra R, Barab{\'a}si AL.
\newblock Historical Comparison of Gender Inequality in Scientific Careers across Countries and Disciplines.
\newblock Proceedings of the National Academy of Sciences. 2020;117(9):4609--4616.
\newblock doi:{10.1073/pnas.1914221117}.

\bibitem{light_gender_2013}
Light R.
\newblock Gender Inequality and the Structure of Occupational Identity: {{The}} Case of Elite Sociological Publication.
\newblock In: Networks, Work and Inequality. vol.~24. Emerald Group Publishing Limited; 2013. p. 239--268.

\bibitem{lynn_rare_2019}
Lynn FB, Noonan MC, Sauder M, Andersson MA.
\newblock A Rare Case of Gender Parity in Academia.
\newblock Social forces. 2019;98(2):518--547.

\bibitem{moody_structure_2004}
Moody J.
\newblock The Structure of a Social Science Collaboration Network: {{Disciplinary}} Cohesion from 1963 to 1999.
\newblock American sociological review. 2004;69(2):213--238.

\bibitem{jadidi_gender_2018}
Jadidi M, Karimi F, Lietz H, Wagner C.
\newblock Gender Disparities in Science? {{Dropout}}, Productivity, Collaborations and Success of Male and Female Computer Scientists.
\newblock Advances in Complex Systems. 2018;21(03n04):1750011.

\bibitem{akbaritabar_gender_2021}
Akbaritabar A, Squazzoni F.
\newblock Gender {{Patterns}} of {{Publication}} in {{Top Sociological Journals}}.
\newblock Science, Technology, \& Human Values. 2021;46(3):555--576.
\newblock doi:{10.1177/0162243920941588}.

\bibitem{teele_gender_2017}
Teele DL, Thelen K.
\newblock Gender in the Journals: {{Publication}} Patterns in Political Science.
\newblock PS: Political Science \& Politics. 2017;50(2):433--447.

\bibitem{cetina_epistemic_1999}
Cetina KK.
\newblock Epistemic {{Cultures}}: {{How}} the {{Sciences Make Knowledge}}.
\newblock Harvard University Press; 1999.

\bibitem{lamont_how_2009}
Lamont M.
\newblock How {{Professors Think}}: {{Inside}} the {{Curious World}} of {{Academic Judgment}}.
\newblock Harvard University Press; 2009.

\bibitem{koppman_joy_2015}
Koppman S, Cain CL, Leahey E.
\newblock The Joy of Science: {{Disciplinary}} Diversity in Emotional Accounts.
\newblock Science, Technology, \& Human Values. 2015;40(1):30--70.

\bibitem{parish_dynamics_2018}
Parish AJ, Boyack KW, Ioannidis JPA.
\newblock Dynamics of Co-Authorship and Productivity across Different Fields of Scientific Research.
\newblock PLOS ONE. 2018;13(1):e0189742.
\newblock doi:{10.1371/journal.pone.0189742}.

\bibitem{lanahan_domino_2016}
Lanahan L, {Graddy-Reed} A, Feldman MP.
\newblock The {{Domino Effects}} of {{Federal Research Funding}}.
\newblock PLOS ONE. 2016;11(6):e0157325.
\newblock doi:{10.1371/journal.pone.0157325}.

\bibitem{lynn_15_2021}
Lynn FB, Espy HW.
\newblock 15. {{Cumulative}} Advantage.
\newblock Research Handbook on Analytical Sociology. 2021; p. 286.

\bibitem{snow_two_2012}
Snow CP.
\newblock The Two Cultures.
\newblock Cambridge University Press; 2012.

\bibitem{bi_four_2023}
Bi HH.
\newblock Four Problems of the H-Index for Assessing the Research Productivity and Impact of Individual Authors.
\newblock Scientometrics. 2023;128(5):2677--2691.
\newblock doi:{10.1007/s11192-022-04323-8}.

\bibitem{berg_producing_2016}
Berg LD, Huijbens EH, Larsen HG.
\newblock Producing Anxiety in the Neoliberal University.
\newblock The Canadian Geographer/le g{\'e}ographe canadien. 2016;60(2):168--180.

\bibitem{mckiernan_metaresearch_2019}
McKiernan EC, Schimanski LA, Nieves CM, Matthias L, Niles MT, Alperin JP.
\newblock Meta-Research: {{Use}} of the Journal Impact Factor in Academic Review, Promotion, and Tenure Evaluations.
\newblock Elife. 2019;8:e47338.

\bibitem{niles_why_2020}
Niles MT, Schimanski LA, McKiernan EC, Alperin JP.
\newblock Why We Publish Where We Do: {{Faculty}} Publishing Values and Their Relationship to Review, Promotion and Tenure Expectations.
\newblock PLOS ONE. 2020;15(3):e0228914.
\newblock doi:{10.1371/journal.pone.0228914}.

\bibitem{forrester_mental_2021}
Forrester N.
\newblock Mental Health of Graduate Students Sorely Overlooked.
\newblock Nature. 2021;595(7865):135--137.
\newblock doi:{10.1038/d41586-021-01751-z}.

\bibitem{braverman_labor_1998}
Braverman H.
\newblock Labor and Monopoly Capital: {{The}} Degradation of Work in the Twentieth Century.
\newblock nyu Press; 1998.

\bibitem{hirsch_index_2005}
Hirsch JE.
\newblock An Index to Quantify an Individual's Scientific Research Output.
\newblock Proceedings of the National Academy of Sciences. 2005;102(46):16569--16572.
\newblock doi:{10.1073/pnas.0507655102}.

\bibitem{barnes_hindex_2017}
Barnes C.
\newblock The H-Index {{Debate}}: {{An Introduction}} for {{Librarians}}.
\newblock The Journal of Academic Librarianship. 2017;43(6):487--494.
\newblock doi:{10.1016/j.acalib.2017.08.013}.

\bibitem{kelly_index_2006}
Kelly CD, Jennions MD.
\newblock The h Index and Career Assessment by Numbers.
\newblock Trends in Ecology \& Evolution. 2006;21(4):167--170.
\newblock doi:{10.1016/j.tree.2006.01.005}.

\bibitem{purvis_index_2006}
Purvis A.
\newblock The h Index: Playing the Numbers Game.
\newblock Trends in Ecology \& Evolution. 2006;21(8):422.
\newblock doi:{10.1016/j.tree.2006.05.014}.

\bibitem{bostonuniversity_boston_2018}
University B.
\newblock Boston {{University School}} of {{Public Health Faculty Handbook}}; 2018.

\bibitem{ohiostateuniversity_department_2014}
University OS.
\newblock Department of {{Surgergy}}, {{Appointments}}, {{Promotion}}, and {{Tenure}}; 2014.

\bibitem{oregonstateuniversity_oregon_2020}
University OS.
\newblock Oregon {{State University College}} of {{Business Faculty Handbook}}; 2020.

\bibitem{ioannidis_citation_2020}
Ioannidis JPA, Boyack KW.
\newblock Citation Metrics for Appraising Scientists: Misuse, Gaming and Proper Use.
\newblock Medical Journal of Australia. 2020;212(6).
\newblock doi:{10.5694/mja2.50493}.

\bibitem{ioannidis_updated_2020}
Ioannidis JPA, Boyack KW, Baas J.
\newblock Updated Science-Wide Author Databases of Standardized Citation Indicators.
\newblock PLOS Biology. 2020;18(10):e3000918.
\newblock doi:{10.1371/journal.pbio.3000918}.

\bibitem{mongeon_journal_2016}
Mongeon P, {Paul-Hus} A.
\newblock The Journal Coverage of {{Web}} of {{Science}} and {{Scopus}}: A Comparative Analysis.
\newblock Scientometrics. 2016;106(1):213--228.
\newblock doi:{10.1007/s11192-015-1765-5}.

\bibitem{singh_journal_2021}
Singh VK, Singh P, Karmakar M, Leta J, Mayr P.
\newblock The Journal Coverage of {{Web}} of {{Science}}, {{Scopus}} and {{Dimensions}}: {{A}} Comparative Analysis.
\newblock Scientometrics. 2021;126(6):5113--5142.

\bibitem{mihaljevic_reflections_2019}
Mihaljevi{\'c} H, Tullney M, Santamar{\'i}a L, Steinfeldt C.
\newblock Reflections on Gender Analyses of Bibliographic Corpora.
\newblock Frontiers in big Data. 2019;2:29.

\bibitem{blevins_jane_2015}
Blevins C, Mullen L.
\newblock Jane, {{John}}... {{Leslie}}? {{A Historical Method}} for {{Algorithmic Gender Prediction}}.
\newblock DHQ: Digital Humanities Quarterly. 2015;9(3).

\bibitem{bell_fixed_2019}
Bell A, Fairbrother M, Jones K.
\newblock Fixed and Random Effects Models: Making an Informed Choice.
\newblock Quality \& quantity. 2019;53:1051--1074.

\bibitem{eubanks_automating_2018}
Eubanks V.
\newblock Automating Inequality: {{How}} High-Tech Tools Profile, Police, and Punish the Poor.
\newblock St. Martin's Press; 2018.

\bibitem{koopman_galton_2023}
Koopman C.
\newblock From {{Galton}}'s {{Pride}} to {{Du Bois}}'s {{Pursuit}}: {{The Formats}} of {{Data-Driven Inequality}}.
\newblock Theory, Culture \& Society. 2023; p. 02632764231162251.

\end{thebibliography}

\section{Supporting Information}\label{supporting-information}

\paragraph*{S1 Table}
\label{id}
{\textbf{Descriptive statistics by discipline.}}


\nolinenumbers

\end{document}
